% !TEX root = ../main.tex

\documentclass[../main.tex]{subfiles}
\begin{document}

As we have mentioned in the previous section, there were various lines of research into thought modeling: automata theory was focused on what problems were possible to model, cybernetics took (analog and biology-based) feedback and self replication as founding pillars, while information theory dedicated itself to information storage and transmission. Instead, DCS was still mostly led by behaviorist views, but cognitive-oriented proposals started to emerge. This trend would continue in 1956, and spike in 1957 with a publication by Noam Chomsky that would change the field.

\subsection{CS}
The most relevant event of 1956 (and quite possibly of the history of AI research) is the Darmouth College Workshop, a sort of convention that connected researchers from diverse fields interested on similar topics. This is also the context in which the term ``Artificial Intelligence" was attached to the field.

\vspace{4pt}
\textbf{Dartmouth College Workshop.}\sidekeyword{Birth of AI}
As we said, at the start of the 1950s thinking machines were being inspected by a few different disciplines; in 1955 John McCarthy, an Assistant Professor at Dartmouth, proposed a conference to organize and fertilize such disciplines. He proposed the name ``Artificial Intelligence" because, unlike today, it was still neutral; Wikipedia reports that avoiding cybernetics was partly due to \enquote*{him potentially having to accept the assertive Norbert Wiener as guru or having to argue with him}. The project was formally proposed in September, by four of those who would become (if they weren't already) prominent researchers in the field: McCarthy himself, Marvin Minsky, Nathaniel Rochester and Claude Shannon. Among the extraordinary attendees we mention: Minsky (who will become very relevant in the next section), Bigelow (co-author of the seminal paper "Behavior, Purpose and Teleology." on cybernetics), Solomonoff (inventor of algorithmic probability), Holland (pioneer of genetic algorithms, was invited but did not end up attending), Ross Ashby (psychiatrist and cybernetics pioneer), McCulloch (who we've already mentioned), Nash (prolific mathematician, also known for his work on game theory), Samuel (creator of what is considered the first AI program, a checkers program) and finally Allen Newell and Paul Simon, who presented their recently completed "Logic Theorist". Although the discussions were not directed, many of the topics would have a long-lasting impact on the field, like the rise of symbolic methods and limiting domains (which would lead to expert systems).

\vspace{4pt}
\textbf{Logic Theorist.}\sidekeyword{Reasoning as search}
The Logic Theorist was created in 1955 by Newell and Simon, helped by the systems programmer John Shaw. In order to prove a theorem, the simplest strategy is to start from the theory's postulates and create new theorems by combining them; then continue by combining every theorem with every postulate and every theorem again, exploring the entire truth spectrum. Although this may seem obvious, this is part of the first important concept introduced by the Logic Theorist: seeing the truth space as a tree, that started with the hypothesis and aimed at the proposition to prove; envisioning \textit{reasoning as search}. Of course, exploring the entire tree is impractical, because of the time it takes to explore the entire truth space (as it grows exponentially); when considered from the point of view of ``modeling the human thought process", this solution would not be useful even if it was practical, because this is not how human theorem-provers work. In order to solve this problem, the Logic Theorist introduced the second important factor: employing \textit{heuristics} to ignore branches that were unlikely to lead to the goal. The last important factor is technical: in order to implement the Logic Theorist, the authors implemented IPL, a programming language that used symbolic \textit{list processing} in the same way as the following, fundamental, Lisp.

\subsection{DCS}
This section, will talk about some of the important findings that seemed difficult to integrate with behaviorism and the important actors behind them.

\vspace{4pt}
\textbf{Miller.}\sidekeyword{Memory study}
George Miller, before becoming one of the founders of cognitive psychology, was of the behaviorist school (although he later wrote of one of his works on language \enquote*{By Skinner’s standards, my book had little or nothing to do with behavior} \parencite{millerCognitiveRevolutionHistorical2003a}). After slowly moving to the cognitive side, driven by similar thinkers (\enquote*{Peter Wason, Nelson Goodman and Noam Chomsky had the most influence on my thinking at that time}). In 1956 he published a paper that had a sizable impact: "The magical number seven, plus or minus two". In it, he observed tht various experimental findings revealed that, on average, human can hold seven items in short-term capacity. We note that it is not the finding that goes against behavioral philosophy and psychology, but the framework in which it is put in general: such attention to mental processes would be irrational, when seen from a behavioral point of view, who disregard mental processes as a whole.

\vspace{4pt}
\textbf{Bruner, Goodnow, Austin, and the basis of cognitive science.}\sidekeyword{Categories as understanding}
Another important book published in 1956 is ``A Study of Thinking", by Bruner, Goodnow and Austin. The book is focused on using categories for concept formation, or how human beings group the world of particulars into classes, together with the results of relevant experiments. Before such experiment on cognition, Bruner had dedicated himself to the study of perception: two relevant studies we report are the one on estimating the sizes of coins or similarly sized wooden sticks (the first were significantly overestimated), and another one on slowing reaction times while playing cards in connection with reversed suit symbols. These two experiments are relevant because of the focus on the internal interpretation of external stimuli. Other foundational ideas of cognitive science, developed in the years following the Miller publication, include the application of the scientific method to human cognition (if anything was to come after behaviorism, it could not avoid its history of scientific ``rigor"), the interest towards information processing and storage, and as we will see in the next paragraph, a degree of possible innateness.

\vspace{4pt}
\textbf{Chomsky and the final departure.}\sidekeyword{Innateness, productions, syntax}
In this last section, we move further than 1956. Nonetheless, it is extremely relevant to the subject discussed, and represents the most decisive blow (in purely historical terms; this document has no psychological authority to express an evaluation of \textit{any} theory) to behaviorism. In 1957, Skinner published ``Verbal Behavior". In it, he describes the controlling elements of verbal behavior, and attempts to form a hypothesis about the behavioral framework with which verbal behavior is to be understood. In it, he uses specific terminology for his analysis, using both existing words and neologisms; in his own words: \enquote*{The emphasis [in Verbal Behavior] is upon an orderly arrangement of well-known facts, in accordance with a formulation of behavior derived from an experimental analysis of a more rigorous sort. The present extension to verbal behavior is thus an exercise in interpretation rather than a quantitative extrapolation of rigorous experimental results} \parencite{skinnerVerbalBehavior1957}.

In the same year, Chomsky proposed a different model for understanding language; in ``Syntactic Structures" he argued two important points that would have a large impact on the field of linguistics:
\begin{enumerate}
    \item \textbf{Syntax vs semantics.} The first point he makes is the clear distinction between syntax and semantics: \enquote*{...such semantic notions as reference, significance, and synonymity played no role in the discussion.}
    \item \textbf{Generative grammars.} His approach to syntax was formal, and followed both his teacher's (Zellig Harris) and notions advanced by Danish linguist Louis Hjelmslev: language was to be understood as a generative grammar, which bound by ``phrase structure rules" (producing new sentences) and ``transformations" (modifying exising sentences).
\end{enumerate}
This seminal paper would soon be interpreted as an argument for a mentalistic, \textit{innate} view of language production. However, this intepretation was not originally put forth in the book itself: \enquote*{[Chomsky's generative system of rules] was more powerful that anything ... psycholinguists had heretofore had at their disposal. [It] was of special interest to these theorists. Many psychologists were quick to attribute generative systems to the minds of speakers and quick to abandon ... Behaviorism} \parencite{steinbergPsycholinguisticsLanguageMind2013}.

Two years later, Chomsky published a scathing (this time both in historical terms and considering the tone of the paper) review  \parencite{chomskyReviewBFSkinner2013} of the book which had a widespread effect of the decline of behaviorism's influence. In it, one of the points he argued was that children are not taught the rules of language, and the amount of input they receive is not sufficient to derive them. This argument would later be called the ``Poverty of the Stimulus" argument, and to this day represent a very controversial issue of linguistics and language acquisition.

In the words of Newmeyer \parencite{newmeyerPoliticsLinguistics1986}:
\begin{quote}
    Chomsky's review has come to be regarded as one of the foundational documents of the discipline of cognitive psychology, and even after the passage of twenty-five years it is considered the most important refutation of behaviorism. Of all his writings, it was the Skinner review which contributed most to spreading his reputation beyond the small circle of professional linguists.
\end{quote}
The review has been criticized by other writers, such as MacCorquodale \parencite{maccorquodaleChomskyReviewSkinner1970}, but its effect cannot be ignored.

\vspace{4pt}
As we have explored in this section, 1956 was both the culmination and start of a cognitively-inspired revolution across the DCS. As behaviorism grew less popular, cognitive findings and research drew more interest. At the same time, one of the very first AI programs was presented, and it tackled a purely symbolical problem with a purely symbolical approach: the trend was clear, and it was pushing towards a cognitive approach.


\end{document}
