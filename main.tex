\documentclass{article}

% package definitions
\usepackage{graphicx}
\graphicspath{img/}
\usepackage[]{biblatex}
\addbibresource{bibliography.bib}
\usepackage{url}
\def\UrlBreaks{\do\/\do-}
\usepackage[breaklinks]{hyperref}
\usepackage{breakurl}
\Urlmuskip=0mu plus 1mu
\usepackage[autostyle]{csquotes}


\usepackage{subfiles} % Best loaded last in the preamble

\title{Cognitive Science and Artificial Intelligence: An Interwoven Approach \\
\large Supervisor: Francesco Bianchini}
\author{Paolo Marzolo}

\begin{document}

\begin{abstract}
    This is where the abstract will go. I guess I'll mention a thing or two about the contents, what I plan to discuss and what my analysis shows.
\end{abstract}

\maketitle
\newpage

\section{Introduction}
\subfile{sections/intro}

\section{Terms and Definitions}
\subfile{sections/glossary}

\section{A History of Influences}
As mentioned in the introduction, our approach will follow the historical sequence of events, although some references or explanations may be anachronistic for clarity. In order to give a general view, we split the histories of these disciplines into broad periods: one for (more or less) every substantial shift in approach and views. Generally, every time period will mention two sides of the story: one of them will focus on DCS, and the other on AI and Computer Science.

\subsection{Landscape before 1950}
\subfile{sections/50_earlier}

\subsection{1956: A Pivotal Year}
\subfile{sections/56_pivotal}

\subsection{1960-1970: Great Promise}
\subfile{sections/60-70}

\subsection{1975-1985: Ashes and Embers}
\subfile{sections/75-85}

\subsection{1987-1993: Bodies as the Key to Minds}
\subfile{sections/87-93}

\subsection{1993-2000: Agents and Cooperation}
\subfile{sections/93-2000}

\subsection{2000-now: Hybrid Systems: New Perspectives}
\subfile{sections/2000-now}

\section{Perception shifts}
Should I merge these two?
\subsection{Symbolism and Connectionism}
\subsection{Symbols and Subsymbols: Collect or Extract}

\section{Conclusion}


\printbibliography


\end{document}