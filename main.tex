\documentclass[oneside,notitlepage]{report}

% package definitions
\usepackage{graphicx}
\graphicspath{img/}
\usepackage[]{biblatex}
\addbibresource{bibliography.bib}
\usepackage{url}
\def\UrlBreaks{\do\/\do-}
\usepackage[breaklinks]{hyperref}
\usepackage{breakurl}
\Urlmuskip=0mu plus 1mu
\usepackage[normalem]{ulem}
\usepackage[autostyle]{csquotes}
\usepackage[english]{babel}
\usepackage{wrapfig}
\usepackage{todonotes} %% won't need it in final version
\usepackage{xcolor}
\usepackage{titling}
\usepackage{float}

\usepackage{subfiles} % Best loaded last in the preamble

\definecolor{lightblue}{HTML}{516eff}
\newcommand{\sidekeyword}[1]{\marginpar{\raggedright\textcolor{lightblue}{\textbf{#1}}}}

\reversemarginpar % use \normalmarginpar to switch back!

\title{Cognitive Science and Artificial Intelligence: An Interwoven Approach \\
\large Supervisor: Francesco Bianchini}
\author{Paolo Marzolo}

\begin{document}
\begin{titlingpage}
    \maketitle
    \begin{abstract}
        This is where the abstract will go. I guess I'll mention a thing or two about the contents, what I plan to discuss and what my analysis shows.
    \end{abstract}
\end{titlingpage}

\tableofcontents
\newpage

\chapter{Introduction}
\subfile{history/intro}

\chapter{Terms and Definitions}
\subfile{history/glossary}

\chapter{A History of Influences}
As mentioned in the introduction, our approach will follow the historical sequence of events, although some references or explanations may be anachronistic for clarity. In order to give a general view, we split the histories of these disciplines into broad periods: one for (more or less) every substantial shift in approach and views. Generally, every time period will mention two sides of the story: one of them will focus on DCS, and the other on AI and Computer Science. At the same time, the two fields are in close relation: because of this, it becomes hard to neatly define the two fields, so some paragraphs will be somewhere in the middle (as they should be!).

\section{Landscape before 1950}
\subfile{history/50_earlier}

\section{1956: A Pivotal Year}
\subfile{history/56_pivotal}

\section{1960-1970: Great Promise}
\subfile{history/60-70}

\section{1970-1985: Symbols and Knowledge}
\subfile{history/70-85}

\section{1987-1993: Bodies as the Key to Minds}
\subfile{history/87-93}

\section{1993-2010: Agents and Cooperation}
\subfile{history/93-2010}

\section{2010-now: Deep Learning and New Perspectives}
\subfile{history/2010-now}

\chapter{Perception shifts}

\subfile{analysis/trends.tex}
\subfile{analysis/symbols.tex}

\chapter{Conclusion}

Throughout this paper, we traced the history of Computer Science, Mathematics, Artificial Intelligence, Psychology, Neuroscience, Philosophy and a host of other disciplines united towards a deeper understanding of intelligence and the mind. All mentioned disciplines have gone through both various phases, but our approach focused on their relationship to symbolic and connectionist models. After the historical perspective, we used the papers we mentioned throughout to trace a cross-historical view of research trends. In a dedicated section, we finally explored recent avenues for symbolic and connectionist integration. This document is meant to serve as both an introductory path through the history of the disciplines, as many insights are to be gained by reminding past proposals in relationship to present paradigms, and as a general picture for how trends move through discoveries, critiques and research.

\appendix
\chapter{Learning and Neural Networks}
\subfile{appendix/learning}

\printbibliography


\end{document}