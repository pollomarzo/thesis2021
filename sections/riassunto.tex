% !TEX root = ../main.tex

\documentclass[../main.tex]{subfiles}
\begin{document}

Come indicato dal \href{https://corsi.unibo.it/laurea/informatica/redazione-tesi-voto-finale}{sito del corso}, includo questa sezione per riassumere brevemente i contenuti della tesi.
\vspace{4pt}
L'obiettivo di questa tesi è analizzare la storia delle Scienze Cognitive e dello studio dell'Intelligenza Artificiale per indentificare come si siano influenzate a vicenda, e come questo abbia portato le due discipline a un'evoluzione parzialmente condivisa. Particolare attenzione è stata fornita a dove, sullo spettro da simbolico a connessionista, cadono i paradigmi e teorie esplorati. Alcune teorie sono spiegate nel dettaglio, per fornire al lettore una chiave di lettura completa.

\section{Struttura del documento}
La tesi è organizzata in quattro capitoli: un breve glossario, la sezione storica, la sezione di analisi e l'appendice. Il glossario è stato incluso per chiarire il significato di alcuni termini usati nel testo; in particolare, abbiamo usato l'acronimo `DCS' (per Descriptive Cognitive Sciences) per indicare la sfera Psicologia, Filosofia, Neuroscienze opponendole a Informatica e Intelligenza Artificiale.

\section{Storia}
La storia affronta lo studio delle discipline sopra menzionate in relazione al concetto di intelligenza secondo una prospettiva, per l'appunto, storica. Le sezioni sono scelte in modo significativo, in corrispondenza con i generali temi di ricerca del periodo. La storia è stata divisa in questo modo:

\begin{itemize}
    \item \textbf{Pre-1950.} Gli argomenti centrali di questa sezione sono le leggi di Boole, la teoria degli automata, la cibernetica, il comportamentismo e alcuni segnali di interesse verso le facoltà orientate cognitivamente. Dalla sezione DCS il messaggio è chiaro: il paradigma di base è il comportamentismo, ma vi sono tracce di interesse per lo studio cognitivo. Per quanto riguarda IA, invece, sono gettate le basi di ciò che verrà considerato in seguito.
    \item \textbf{1956.} Il 1956 è trattato a sé stante perché individuato come anno di svolta. Dopo aver considerato il Workshop del college di Dartmouth e i principali attori, l'attenzione è spostata sul Logic Theorist e l'introduzione del concetto di \textit{ragionamento come ricerca nell'albero delle scelte}. Seguono ulteriori ricerche cognitive e la critica di Chomsky a ``Verbal Behavior''.
    \item \textbf{1960-70.} In questo ventennio la ricerca in IA si solidifica chiaramente nel paradigma simbolico. Sono esplorati il General Problem Solver, il programma di dama di Samuel e alcuni dei successi simbolici. In DCS, sono evidenziate le ricerche sui neuroni e lo spostamento del comportamentismo nella sfera pratico-terapeutica.
    \item \textbf{1970-85.} In questi quindici anni, è evidenziato l'abbandono dei modelli connessionisti e l'integrazione delle conoscenze di settore nei programmi di IA. Inoltre, è affrontata la teoria computazionale della mente.
    \item \textbf{1987-93.} Questa sezione evidenzia il successo dei sistemi esperti, e il successivo crollo economico durante la realizzazione del loro costo operazionale. L'intero spettro disciplinare si sposta verso il lato connessionista, seguendo il paradigma robotico di ``abbassare la mente nel corpo''. Attore fondamentale è Brooks.
    \item \textbf{1993-2010.} Avvicinandoci ai tempi recenti, i temi diventano più familiari alle orecchie moderne. Il più grosso impatto è attribuito alla disponibilità di dataset di dimensioni massicce. Sul lato DCS, sono menzionati alcuni strumenti neuroscientifici importanti e la distinzione tra teorie unificatrici e teorie espansive.
    \item \textbf{2010-oggi.} In quest'ultima sezione sono affrontati gli ultimi sviluppi per quanto riguarda il linguaggio naturale, l'analisi di immagini e gli approcci neuro-simbolici. Viene evidenziato un tentativo unificatore di teorie della mente e alcune scoperte in neuroscienza.
\end{itemize}

\section{Analisi}
In questa sezione, vengono presentate due riflessioni non distribuibili completamente all'interno della visione storica: le tendenze tra simbolismo e connessionismo e la relazione tra rappresentazioni simboliche e subsimboliche.
\subsection{Tendenze}
Con un grafico qualitativo, vengono evidenziati gli spostamenti nello spettro notati durante la prospettiva storica. I due grossi rami sono distinti da colore diverso, mentre i due assi rappresentano rispettivamente l'asse temporale e la posizione (approssimativa) del paradigma o articolo in questione sullo spettro sopra citato.
\subsection{Simboli, subsimboli e approcci integrativi}
Quest'ultima sezione viene usata per un'introduzione al tema delle rappresentazioni tra simboli (anche dette localizzate) e subsimboli (rappresentazioni distribuite). Viene poi descritta la tassonomia di Kautz per la classificazione di approcci ibridi.

\section{Contenuti dell'appendice}
Le due appendici servono per questo riassunto finale e per una parentesi importante su alcuni dettagli implementativi delle reti neurali. Questo permette al testo di dare per assodati concetti specifici come reti ricorrenti o convoluzionali, e di poter descrivere le architetture senza necessità di spiegazione approfondita. Sono presenti link a ulteriori fonti e risorse.

\end{document}