% !TEX root = ../main.tex

\documentclass[../main.tex]{subfiles}
\begin{document}

Before definining our glossary, it is important to understand the reasoning behind why we chose to include it: when discussing researchers' understanding of human thought, it is nearly impossible to avoid using terms that have a strong past history. As an example, "thought" could already be considered too far from a behaviorist point of view. A further example is a recent discussion that took place after a somewhat controversial paper by Nunez was published \cite{nunezWhatHappenedCognitive2019}, questioning the multidisciplinarity of Cognitive Science as a discipline (and journal) and declaring \enquote{The prospect launched by the cognitive revolution of a unified and coherent interdisciplinary seamless cognitive science did not materialize}.


\vspace{5pt}
\textbf{Cognitive Science.} As we will see in following sections, saying "definitions of Cognitive Science have evolved throughout the years" would be a massive understatement. (“Thinking can best be understood in terms of representational structures in the mind and computational procedures that operate on those structures”). Its multidisciplinary nature is uncontested from what the International Encyclopedia of Social \& Behavioral Sciences \cite{InternationalEncyclopediaSocial} reports \enquote*{ may have been the first published use of the term cognitive science}:
\begin{quote}
    ‘The concerted efforts of a number of people from ... linguistics, artificial intelligence, and psychology may be creating a new field: cognitive science’
\end{quote}. Even the "essential original features" identified by Gardner in 1987 \cite{gardnerMindNewScience1987} (summarized here as (1) necessity to speak about mental representation as a separate layer of analysis from the biological, (2) faith that the computer is central to the understanding of the human mind and (3) de-emphasizing factors such as emotions or cultural factors) would be completely or partially thrown out by contemporary scholars.

In a more recent publication\cite{bodenMindMachineHistory2008}, Cognitive Science is characterized as
\begin{quote}
    The field would be better defined as the study of ‘mind as machine’ ... More precisely, cognitive science is the interdisciplinary study of mind, informed by theoretical concepts drawn from computer science and control theory.
\end{quote}.

Not only was its definition cloudy and unstable (“cognitive science is ... a perspective, rather than a discipline in any conventional sense" \cite{sheehyCognitiveScience1995}), but as Nunez points out its disciplines have varied wildly in which ones they are and how represented they are in the Cognitive Science enterprise. Because of the reasons outlined here, far removed form the subject of this document, we will avoid using the term "Cognitive Science", and prefer the acronym "DCS".

\vspace{5pt}
\textbf{Descriptive Cognitive Sciences (DCS)}. As we mentioned, the disciplines which make up Cognitive Science are not only multiple, but subject to interpretation as well. Since the nature of this work is to compare it to the history of Artificial Intelligence, we will from this point on use the acronym "DCS", for Descriptive Cognitive Sciences, as an alternate approach to the Constructive one taken by Artificial Intelligence researchers. This is not to say that a psychologist cannot take a constructive approach to the explanation of consciousness: the only reason we chose this is because we found it to be an intuitive use of the term.

\vspace{5pt}
\textbf{Mind.} Once again, although we take notice of the history of the term, we have to select a few terms to use in our language. Hereafter, we consider the mind as the non-physical correlate of human brains: "the complex of faculties involved in perceiving, remembering, considering, evaluating, and deciding. Mind is in some sense reflected in such occurrences as sensations, perceptions, emotions, memory, desires, various types of reasoning, motives, choices, traits of personality, and the unconscious."\cite{Mind}.

\vspace{5pt}
\textbf{whatever else will come up}


\end{document}
