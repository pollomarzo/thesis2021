% !TEX root = ../main.tex

\documentclass[../main.tex]{subfiles}
\graphicspath{{\subfix{../img/}}}
\begin{document}

Before definining our glossary, it is important to understand the reasoning behind why we chose to include it: when discussing researchers' understanding of human thought, it is nearly impossible to avoid using terms that have a strong past history. As an example, "thought" could already be considered too far from a behaviorist point of view. A further example is a recent discussion that took place after a somewhat controversial paper by Nunez was published \cite{nunezWhatHappenedCognitive2019}, questioning the multidisciplinarity of Cognitive Science as a discipline (and journal) and declaring \enquote{The prospect launched by the cognitive revolution of a unified and coherent interdisciplinary seamless cognitive science did not materialize}.


\vspace{5pt}
\textbf{Cognitive Science.} As mentioned in the Nunez paper... STUFF. Because of this, the disciplines which make up Cognitive Science are not only multiple, but subject to interpretation as well. Since the nature of this work is to compare it to the history of Artificial Intelligence, we will from this point on use the acronym "DCS", for Descriptive Cognitive Sciences, as an alternate approach to the Constructive one taken by Artificial Intelligence researchers. This is not to say that a psychologist cannot take a constructive approach to the explanation of consciousness: the only reason we chose this is because we found it to be an intuitive use of the term.

\vspace{5pt}
\textbf{Mind.} Once again, although we take notice of the history of the term, we have to select a few terms to use in our language. Hereafter, we consider the mind as the non-physical correlate of human brains: "the complex of faculties involved in perceiving, remembering, considering, evaluating, and deciding. Mind is in some sense reflected in such occurrences as sensations, perceptions, emotions, memory, desires, various types of reasoning, motives, choices, traits of personality, and the unconscious."\cite{Mind}.

\vspace{5pt}
\textbf{whatever else will come up}


\end{document}
