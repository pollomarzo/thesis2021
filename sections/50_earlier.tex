% !TEX root = ../main.tex

\documentclass[../main.tex]{subfiles}
\graphicspath{{\subfix{../img/}}}

\begin{document}

Although the official birth of the "Cognitive Science" institutions is in the late 1970s, reasoning about thought has been a staple in philosophical research for centuries. Because of the scope of this document, we will focus on a few important concepts, and use them to set the stage for the first large shift of ideas.

\subsubsection{Mathematics and Computer Science}
Some of the most relevant contributions to the "reasoning as a process" come from Mathematics and what would  later become Theoretical Computer Science. We will outline some of them here.

\textbf{Boole's Laws of Thought and Boolean Algebra.} To avoid going too deep in mathematical concepts for our purposes, we can think of Boolean algebra as the branch of algebra where the variables can be either true or false (1 and 0), and the main operations on its variables are conjuction (and, $\wedge$), disjunction (or, $\vee$), negation (not, $\neg$). Through these, logical operations can be described.
In "An Investigation of the Laws of Thought on Which are Founded the Mathematical Theories of Logic and Probabilities", one of the author's two monographs on algebraic logic, George Boole, then mathematics professor in Ireland, introduces Boole's algebra as an extension to Aristotle's logic. In it, Boole provides Aristotle's algebra with mathematical foundations, and expands it from two-term to any-term. Boole's algebra differs from modern Boolean algebra (in Boole's algebra \textit{uninterpretable} terms exist) and cannot be inteprepted as set operations; still, its introduction marks a step towards the formalization of laws of thought and a possible bridge between mathematical research and thinking processes. Boolean algebra would instead be developed by Boole's successors (Jevons, Peirce, Schroder and Huntington in particular); this work allows boolean algebra to now be defined by the Stanford Encyclopedia as
\begin{quote}
    the algebra of two-valued logic with only sentential connectives, or equivalently of algebras of sets under union and complementation.
\end{quote}

\textbf{Cybernetics.}

- Boole "Laws of Thought"
- automata theory
- cybernetics
- information theory

topics:
- behaviorism
- gestalt?
- Vygotsky-Luria?
- Several psychologists who later pioneered a more cognitive approach, including Miller, Ulric Neisser, and Donald Norman, received their training in S. S. Stevens’s Psycho-acoustic Laboratory at Harvard
- simplest mcCullough-Pitts neuron is 1943!


\end{document}
